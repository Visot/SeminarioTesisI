\documentclass[12pt,spanish, singlespacing,]{MastersDoctoralThesis}
\usepackage[utf8]{inputenc} 
\usepackage[T1]{fontenc} 
\usepackage[none]{hyphenat}
\usepackage{palatino} 
\usepackage{acronym}
\usepackage{amsmath}
\usepackage{float}
\usepackage[caption = false]{subfig}
\spacing{1.5}
\usepackage[acronym,nomain]{glossaries}
%\usepackage[backend=bibtex,style=authoryear,natbib=true]{biblatex}
%\addbibresource{main.bib}
\usepackage[autostyle=true]{csquotes}
\usepackage{tikz}
\usetikzlibrary{shapes.geometric, arrows}
\usepackage{smartdiagram}
\usepackage{afterpage}
\tikzstyle{startstop} = [rectangle, rounded corners, minimum width=4cm, minimum height=1.5cm,text centered, draw=black, fill=blue!30]
\tikzstyle{arrow} = [thick,->,>=stealth]
\newcommand\blankpage{%
    \null
    \thispagestyle{empty}%
    \addtocounter{page}{0}%
    \newpage}
    
\usepackage{listings}
\renewcommand{\lstlistingname}{Código}
\renewcommand\lstlistlistingname{Índice de Código}

\usepackage{float}
\usepackage{url}
\makeatletter
\g@addto@macro{\UrlBreaks}{\UrlOrds}
\makeatother

\thesistitle{}
\supervisor{} 
\examiner{} 
\degree{} 
\author{} 
\subject{}
\keywords{} 
\university{\href{http://www.uni.edu.pe/}{Universidad Nacional de Ingenier\'ia}} 
\department{\href{http://fc.uni.edu.pe/fc/index.php/escuelas/ciencia-de-la-computacion}{}} 
\faculty{\href{http://fc.uni.edu.pe/fc/}{}}
\hypersetup{pdftitle=\ttitle} 
\hypersetup{pdfauthor=\authorname}
\hypersetup{pdfkeywords=\keywordnames} 
\sloppy
\decimalpoint
\begin{document}
\frontmatter 
\pagestyle{plain} 
\begin{titlepage}
\begin{center}
%\textsc{\huge \univname}\\[0.5cm]

\begin{figure}[h]
\centering
\includegraphics[width=0.3\textwidth]{Figures/log_uni.png}
\end{figure}

\textsc{\huge \univname}\\[0.3cm]
\textsc{\Large Facultad de Ciencias}\\[0.2cm]
\textsc{\large Escuela Profesional de Ciencia de la Computaci\'on}\\[2cm]
\textsc{\LARGE \textit{Métodos de optimización de la gradiente de descenso en una red neuronal convolucional}}\\[2cm] 
{\Large \textbf{SEMINARIO DE TESIS 1}}
{\huge \bfseries \ttitle}\\[2cm] 
% Thesis title
%\bigskip
%\HRule \\[0.8cm] % Horizontal line

%\begin{minipage}{1.5\textwidth}
%\begin{flushleft} \large
\bigskip
\bigskip
\large\emph{Autor: Víctor Jesús Sotelo Chico}
{\authorname}\\ 
\large\emph{Asesor: Víctor Melchor Espinoza}
{\supname} 
%\end{flushleft}
%\end{minipage}
%\\[2cm]
\\[1cm]
{\large Junio, 2018}\\[4cm] 
 
\vfill
\end{center}
\end{titlepage}
\afterpage{\blankpage}
%\cleardoublepage
%\renewcommand{\abstractname}{Abstract}
\begin{abstract}
\addchaptertocentry{\abstractname}
En las últimas décadas el campo de la inteligencia artificial se ha desarrollado rápidamente. Temas como el reconocimiento de imágenes han sido estudiado por mucho tiempo. Actualmente este campo requiere realizar un gran números de cálculos para entrenar redes neuronales que sean capaces de distinguir y clasificar distintos objetos contenidos en imágenes. Incluso este proceso puede tardar más dependiendo del tamaño del dataset. Por lo cual surge la necesidad de encontrar métodos que permitan acelerar el proceso de entrenamiento de las redes neuronales.\\
Por tal motivo presente seminario buscar lograr un mayor entendimiento de métodos para acelerar el proceso de entrenamiento de una red neuronal convolucional, basándonos en la teoría de redes neuronales y usando como herramienta la librería tensorflow, esta nos permite un uso controlado de las métodos de optimización.

\textbf{KEYWORDS}:  Dataset, Métodos Adaptativos, CNN, SGD, Optimizadores.
\end{abstract}\hspace{10pt}

\keywords{agasga}


\afterpage{\blankpage}
\tableofcontents
%\afterpage{\blankpage}
\listoffigures
%\afterpage{\blankpage}
%\listoftables
%\afterpage{\blankpage}
%\listoftables 
%\lstlistoflistings
%\afterpage{\blankpage}
\newpage

\begin{center}
{\huge Índice de Acrónimos}\\[2cm]
\end{center}
\bigskip
\begin{tabular}{ l c l }
\textbf{k-nn}& & k- nearest neighbors\\
\textbf{SVM}& & Super Vector Machine\\
\textbf{SVC}& & Super Vector Regression\\
\textbf{SVR}& & Super Vector Classification\\
\textbf{SGD} & & Stochastic gradient descent\\
\textbf{DNN} & & Deep Neural Network\\
\textbf{CNN} & & Convolutional Neural Network\\
\textbf{ETC} & & Etcétera \\

\end{tabular}
\afterpage{\blankpage}

\newpage
\begin{center}
{\huge \textit{Agradecimientos}}\\[1.5cm]
\end{center}

Agradezco a mis padres por todo el apoyo incondicional durante todos estos años de estudio, a mis compañeros de clase por el apoyo brindado durante el tiempo de estudio y a mi asesor por ayudarme en este seminario.



\afterpage{\blankpage}

\mainmatter 
\pagestyle{thesis}
\chapter{Introducción}
Escribir un texto de un o dos párrafo(s) máximo de 10 líneas con una introducción al capítulo


\section{Motivación}
Que es lo que te ha motivado para realizar la tesis en esta temática y los aspectos más esenciales que pensabas obtener de ella...

Este punto podrá ser de 4 páginas máximo. Es una de las partes más importantes que introduce al lector en el trabajo en tu tesis, por lo que debe estar muy bien redactado y estructurado.

\section{Objetivos}

El objetivo de este seminario es el de ... PONER OBJETIVO GENERAL .

Especificamente, los objetivos de este trabajo con respecto al sistema son:

\begin{itemize}
\item[•] OBJETIVO ESPECÍFICO 1.
\item[•] OBJETIVO ESPECÍFICO 2.
\item[•] OBJETIVO ESPECÍFICO 3.


\end{itemize}

Y los objetivos con respecto a las competencias académicas desplegadas en el trabajo son:
\begin{itemize}
\item[•] OBJETIVO COMPETENCIA 1.
\item[•] OBJETIVO COMPETENCIA 2.
\item[•] OBJETIVO COMPETENCIA 3.

\end{itemize}

\section{Estructura del Seminario}

Para brindrar al lector una idea global del contenido de este trabajo, a continuación se hace una breve descripción del propósito de cada capítulo presente en este seminario de tesis.
\begin{itemize}

\item \textbf{Introducción:} \\
En este capítulo introductorio se comenta sobre las motivaciones ...

\item \textbf{Estado del Arte:} \\
En este capítulo se establece el entorno ...

\item \textbf{Metodología y Herramientas:} \\
....

\item \textbf{...} \\
Un pequeño texto más de ese capítulo(s) en concreto.

\item \textbf{Conclusiones y Trabajo a Futuro:}\\
En este capítulo se exponen las conclusiones obtenididas de este trabajo. Adicionalmente, se proponen trabajos a futuro para la implementación de el servidor y la aplicación web del sistema.
\end{itemize}

NOTA: RECUERDE QUE ES COMO UN LIBRO TODO CAPÍTULO NUEVO COMENZARÁ CON PÁGINA IMPAR NUNCA PAR



%\newpage
%$\ $
%\thispagestyle{empty} % para que no se numere esta pagina
\chapter{Estado del Arte}
En este capítulo describiremos anteriores investigaciones de machine learning, además de sus aplicaciones. Además veremos algunas investigaciones GPU como un modo de obtener un mejor rendimiento y nos enfocaremos principalmente en los estudios de los métodos de optimización.

También mostraremos investigaciones referentes a deep learning exclusivamente nos enfocaremos a Convolutional Neural Network(CNN) ya que son parte del tema de estudio en este seminario. 

%---Escribir un texto de un o dos párrafo(s) máximo de 10 líneas con una introducción al capítulo 

%---El capítulo estado del arte es tanto o más importante que la tesis en sí. En el se debe especificar que desarrollo relacionado a tu tesis existe ya a nivel global y en que se diferencia tu trabajo de ellos. Por lo tanto un análisis exhaustivo de la especialidad y de los trabajos previos es tanto o más importante que el trabajo en sí, ya que indica un alto conocimiento de la materia si está bien estudiado.

%---En este capítulo van a ir muchas citas \cite{Wan09} de trabajos pero sobre todo de artículos científico, haga un buen estudio del arte \cite{Shuo10,Feldmann03}


\section{GPU computing}
El uso de las GPU's han permitido lograr aplicaciones que antes podriamos imaginar que eran imposibles debido a su largo tiempo de ejecución. Hoy en día las GPU's son altamentes usadas debido que cuentan con cientos de núcleos de procesadores en paralelo que permiten resolver rápidamente los problemas que son altamente paralelizables.
\subsection{The GPU computing Era}
	En artículo se enfoca principalmente en describir la evolución que sufrieron las arquitecturas de GPU's, además de mostrar la importancia del uso de las GPU's para un mayor rendimiento y eficiencia que antes hubiesen sido consideradas imposibles de	bido al alto tiempo de ejecución que requerian. Además nos muestra que la escalabilidad es la principal característica que ha permito que las GPU's aumenten su paralelismo y redimiento.
\section{Machine learning}
El uso de machine learning representa una gran ventaja para empresas que manejan gran cantidad de datos debido a que permiten descubrir patrones y analizar los datos.

\subsection{Uso de redes neuronales para encontrar el rendimiento de una GPU}
En la actualidad existen empresas dedicadas a la creación de GPU's, en el proceso una parte fundamental es la verificación del rendimiento de las GPU's actualmente existen simuladores conocidos como GPGPU-SIM que permiten estimaciones precisas pero estos poseen algunas dificultades como el tiempo empleado en configurarlos en base al hardware real además que este proceso esta propenso a errores. Un equipo de equipo conformado por investigadores de AMD y The University of Texas at Austin, quienes propusieron el uso de redes neuronales para predecir el rendimiento.
\section{Deep Learning}
Dentro del área de machine learning  encontramos deep learning o aprendizaje profundo el cual consiste en un conjunto de algoritmos que modela abstracciones de alto nivel.

\subsection{Deep Machine Learning - A New Frontier in Artificial Intelligence}
Este trabajo de investigación fue realizado por investigadores oak Ridge National Laboratory y University of tennessee, el objetivo principal de este trabajo fue presentarnos el aprendizaje profundo como un camino para la imitación del cerebro humano y sus principales cualidad como el reconocimientos de objetos, rostros, etc.

Además de mostrarnos las aplicaciones del aprendizaje profundo, como : analísis de documentos, detección de voz, rostro, procesamiento natural del lenguaje, etc.

Actualmente existen algunas empresas privadas que apoyan el campo de deep learning con el objetivo de buscar sus aplicaciones comerciales entre estas empresas tenemos: Numenta y Binatix.
\section{Métodos de optimización}
El campo de machine learning continuamente evoluciona y con esta evolución surgen nuevas necesidades al trabajar con grandes conjuntos de datos se buscan cada vez obtener buenos resultados sin afectar el rendimiento. Una forma de lograr esto es mediante el uso de algoritmos de optimización.
\subsection{Neural Network Optimization Algorithms: A comparison study based on TensorFlow}
Vadim Smolyakov realizo un estudio comparativo de diversos optimizadores entre los cuales se encuentran el método de gradiente de descenso estocástica, Nesterov Momentum,RMSProp y Adam. Se realizo una prueba comparativa con una arquitectura simple de CNN usando el conjunto de datos del MNIST. \textquotedblleft Se comparo diferentes optimizadores y se obtuvo que SGD con Nesterov y Adam producen mejores resultados en el entrenamiento de una CNN simple usando tensorflow \textquotedblright \cite{Optimization}
\subsection{On Optimization Methods for Deep Learning}
Un equipo de la universidad de standford realizó una pruebas con el objetivos de encontrar métodos adecuados para un entrenamiento en deep learning. El equipo se percato de lo común que resulta el uso de Gradiente de descenso estocástica o SGD por sus siglas en inglés en deep learning . Se realizaron pruebas con otros métodos de optimización como la gradiente conjugada y Limited memmory BFGS(L-BFGS) los cuales permitieron acelerar el proceso de entrenamiento de algoritmos de deep learning mostrando en su mayoría mejores resultados que el SGD. \textquotedblleft Usando L-BFGS el modelo CNN alcanza el 0.69\%  en el estandar del MNIST dataset. \textquotedblright
\section{Conclusiones}
%Hemos visto la necesidad....
A medida que tratamos muchos problemas vemos la necesidad de encontrar optimizadores adecuados para los tipos de problemas. En el área de deep learning comumente se trabaja en el campo de reconocimiento de imágenes a pesar de la mejorar mediante el uso de GPU's este tipo de problemas necesitan métodos óptimos para obtener una mejor performance.
%Poner unas conclusiones del capítulo y lo más importante, donde se enfoca tu trabajo y lo que se diferenncia del resto


%\newpage
$\ $
%\thispagestyle{empty} % para que no se numere esta pagina
\chapter{OTRO CAPITULO}

ADSFASDFAS

\section{Metodología}
ASDFASDFASDFA

\subsection{SCRUM}

ASDFASDFASDFA


\subsubsection*{Características}

\begin{figure}[H]
\centering
\includegraphics[width=0.9\textwidth]{Figures/mvc.png}
\caption{Modelo-Vista-Conrolador}
\label{MVC}
\end{figure}


\afterpage{\blankpage}
%\newpage
$\ $
%\thispagestyle{empty} % para que no se numere esta pagina
\chapter{Algoritmos de optimización una Red Neuronal convolucional}
En este capítulo se detallarán los algoritmos de optimización de Aprendizaje automático y principalmente se enfocará en aquellos aplicados a las redes neuronales convolucionales.

\section{Redes Neuronales Convolucionales}
Las CNN son un tipo de redes neuronales especiales para procesar data del tipo de topología grid. El termino convolucional hace referencia a la operación lineal matemática usada. Las redes neuronales convolucionales usan esta operación para aprender de las características de alto orden en la data.
La primera CNN fue creada por Yann LeCun. Entre sus usos más comunes tenemos el reconocimiento de imágenes y lenguaje natural.\\
Las redes neuronales convolucionales fueron inspiradas en la corteza visuales de los animales. Las celulas de la corteza visual principalmente se activan para realizar tareas como el reconocimiento de patrones.

\subsection{Estructura de una imagen}
Debido a que las redes neuronales convoluciones trabajan principalmente con imagenes es importante conocer cual es la estructura de una imagen y como la computadora comprende utiliza esta información.
Las imágenes están constituidas por la sucesión de pixeles podemos entender el pixel como la menor unidad homogenea en color de una imagen digital. Teniendo este concepto la información de una imagen puede dividirse de la siguiente forma:
\begin{itemize}
	\item Width: El ancho de la imagen medido en pixeles
	\item Height: El alto de la imagen medida en pixeles.
	\item Canales RGB: Estos canales contiene la información de los colores y profundidad de una imagen. Este canal guarda la información en tres canales Red, Green y Blue.
\end{itemize}
Teniendo en cuenta esta forma de guardar una podemos resaltar el porque de la ventaja de usar Redes convolucionales en lugar de usar una red neuronal multicapas.\\ Las redes multicapas toman un vector de una dimensión como entrada si quisieramos entrar una red multicapas con imagenes de 32x32 pixeles y con 3 canales RGB necesitariamos crear 3072 pesos ($w_{i}$) para una sola neurona en la capa oculta. La creación hace que la tarea resulte complicada usando redes multicapas.
\subsection{Arquitectura General de CNN}

\subsubsection{Input layer}
Esta capa es donde se carga y almacena la información de las imágenes para procesarlas en la red. Esta información contiene detalles de ancho, alto y número de canales de imagen.

\subsubsection{Convolutional layers}
Son una capa importante en el diseño de las CNN's, las capas convolucionales transformarán la entrada de la data usando las conexiones de las neuronas de las capas anteriores. La capa calculará el producto punto entre la región de la neurona de la capa de entrada y los pesos a los que están colocados localmente en la capa de salida. Esta salida tendrá la misma dimensión de espacios o una dimensión menor.

Para entender más a fondo debemos definir la operación de convolución.  La convolución es una operación matematica que describe una regla de como fusionar 2 conjuntos de información.\textquotedblleft Esta operación tiene importancia en campos como la matematica y la física debido que permite definir un puente entre el domino del espacio/tiempo y el dominio de la frecuencias a través del uso de transformada de fourier. Toma la entrada un entrada, aplica un kernel de convolución y nos da un mapa de características como salida \textquotedblright \cite{book1} .\\
Las convoluciones son usadas principalmente como un detector de características cuyas entradas principalmente son la capa de entrada u otra convolución.
En la figura 4.1 observamos la operación de convolución que por medio del uso de un kernel o filtro de convolución extrae características de la data por ejemplo detalles como bordes de una imagen. Haciendo analogía con los pesos en las redes neuronales convencionales, las redes poseen el filtro o kernel lo cual es beneficioso ya que no se debe definir un peso para cada neurona.
\begin{figure}[H]
	\centering
	\includegraphics[width=0.7\textwidth]{Figures/convolucion.jpeg}
	\caption{Operacion de convolución \\ Fuente:  \href{http://openresearch.ai/t/network-in-network/39}{\textit{www.openresearch.ai}}}
	\label{convolucion}
\end{figure} 

\textbf{Componentes de la capa de convolución}
Las capas convolucionales poseen paramétros e hiperparametros. La gradiente de descenso se usa principalmente para entrenar los parametros de modo que las clases sean consistentes con las etiquetas en el conjunto de entrenamiento. Entre estos parámetros tenemos:
\textbf{Filtros}

Los filtros son una función que posee ancho(width) y alto (height) más pequeños que la entrada. Los filtros son aplicados a través de  del ancho y alto de la entrada. También pueden ser aplicados a la profundidad.
\subsubsection{Classification layers}

\subsection{Arquitecturas conocidas}
\section{Métodos de optimización}
Los métodos de optimización dentro del campo de deep learning son muy importantes debido a que esistentes gran cantidad de parámetros es 

\subsection{Gradiente de descenso}
La gradiente de descenso es un algoritmo común para optimizar redes neuronales. La gradiente de descenso es una forma de minimizar la función de costo $J(\theta)$ para metrizada por los parámetros $\theta \in\Re^{d}$ actualizando los parámetros en la dirección opuesta a la gradiente de la función objetivo en este caso a nuestra función de costo $\nabla_{\theta} J(\theta)$
Dentro de la gradiente de descenso podemos diferenciar 3 variantes de acuerdo al la cantidad de datos que se usan para calcular la gradiente de la función objetivo entre estas variantes tenemos a:\\
\subsubsection{Batch gradient descent}
Esta variante calcula la gradiente de descenso de la función de costo con respecto a un parámetro $\theta$, para todo el conjunto de datos. En la ecuación 4.1 podemos observar la actualización que se dará para cada ejecución. $\eta$ representa la taza o tamaño de los pasos para encontrar el mínimo local.
\begin{equation}
\label{bgds}
\begin{aligned}
\theta &= \theta - \eta \nabla_{\theta} J(\theta)
\end{aligned}
\end{equation}
La ecuación 4.1 asegura la convergencia para mínimo global en una supercifie convexa y mínimo local para una supercifie no convexa. Entre las dificultades de este método tenemos que puede llegar a ser lento y que esta limitado por la cantidad de datos ya que esta puede superar a la memoria del computador.	
\subsubsection{Stochastic gradient descent}
A diferencia del método anterior las actualización se realizan para cada ejemplo de entrenamiento de $(x^{i},y^{i})$ de esta manera se evitan problemas como la generación de redundancia debido a que se realiza una actualización por ejemplo de entrenamiento.
\begin{equation}
\label{sgds}
\begin{aligned}
\theta &= \theta - \eta \nabla_{\theta} J(\theta,x^{i},y^{i})
\end{aligned}
\end{equation}
En la figura 4.2 vemos que la función de costo en SGD fluctúa demasiado esto podría representar un problema pero al contrario de la figura representa que el método SGD es capaz de saltar de un mínimo local con lo cual puede encontrar mínimos locales potencialmente mejores.

\begin{figure}[H]
	\centering
	\includegraphics[width=0.5\textwidth]{Figures/sgd.png}
	\caption{Función costo en SGD \\ Fuente:  \href{https://www.doc.ic.ac.uk/~js4416/163/website/neural-networks/optimisers.html}{\textit{www.doc.ic.ac.uk}}}
	\label{funcion costo}
\end{figure} 

\subsubsection{Mini-batch gradient descent}
Este método pude verse como una mezcla de los 2 métodos anteriores, en lugar de aplicarlo para un conjunto entero de datos, los datos se particionan en pequeños conjuntos o mini batches.
Este método nos permite reducir la varianza de las actualizaciones de los parámetros lo cual nos permite una convergencia más estable. El tamaño de los mini-batches oscilan entre 50-250 y varían de acuerdo a su aplicación.

\begin{equation}
\label{mbgds}
\begin{aligned}
\theta &= \theta - \eta \nabla_{\theta} J(\theta,x^{i:i+n},y^{i:i+n})
\end{aligned}
\end{equation}



\subsection{Optimizadores}
En la siguiente sección analizaremos algunos optimizadores que acelerarán el proceso de gradiente de descenso.
\subsubsection{Momentum}
Las SGD tienen problemas para desplazarse en áreas con donde la superficie se curva más en una dimensión que en otra, estos lugares son los alrededores de los óptimos locales. En este escenario la SGD oscilará en la curvatura y descenderá lentamente hacia el óptimo como se muestra en la figura 4.3.
\begin{figure}[H]
	\centering
	\includegraphics[width=0.5\textwidth]{Figures/momentum1.png}
	\caption{Actualización sin momentum \\ Fuente:  \href{https://www.doc.ic.ac.uk/~js4416/163/website/neural-networks/optimisers.html}{\textit{www.doc.ic.ac.uk}}}
	\label{momentum1}
\end{figure}
El momentum es un método que ayuda a la SGD a acelerar en la dirección correcta, mientras evitas las oscilaciones. El momentum lográ esto añadiendo una fracción $\gamma$ del vector de actualización pasado al vector presente tal como se muestra en las ecuaciones 4.4. Un valor comunmente elegido de $\gamma =0.9 $, en las actualización el valor del momentum aumenta para dimensiones cuyos gradientes apuntan en la misma dirección y disminuye para dimensional en la que la gradiente cambia de dirección. Esto nos asegura que tendremos una convergencia más rápida con una oscilación reducida.En la figura 4.4 se observa gráficamente la aceleración de la convergencia en la SGD.

\begin{equation}
\label{mbgds}
\begin{aligned}
\nu_{t}&=\gamma \nu_{t-1} +  \eta \nabla_{\theta} J(\theta)\\
\theta &= \theta -\nu_{t}
\end{aligned}
\end{equation}

\begin{figure}[H]
	\centering
	\includegraphics[width=0.5\textwidth]{Figures/momentum2.png}
	\caption{Actualización con momentum \\ Fuente:  \href{https://www.doc.ic.ac.uk/~js4416/163/website/neural-networks/optimisers.html}{\textit{www.doc.ic.ac.uk}}}
	\label{momentum2 }
\end{figure}

\subsubsection{Nesterov accelerated gradient}
Este método en el que nuestro descenso sea más controlado ya que reduce la velocidad antes de volver a subir una pendiente. En momentum usamos el término $\gamma \nu_{t-1}$ para mover los parámetros de $\theta$. Al calcular el valor de $\theta - \gamma \nu_{t-1}$ nos da una aproximación de donde se encontrá la siguiente posición de los parámetros. De esta forma no calculamos la gradiente en el parámetro $\theta$ actual sino que se calcula en una posición futura aproximada.




\begin{equation}
\label{mbgds}
\begin{aligned}
\nu_{t}&=\gamma \nu_{t-1} + \eta \nabla_{\theta} J(\theta- \gamma \nu_{t-1})\\
\theta &= \theta -\nu_{t}
\end{aligned}
\end{equation}

En la figura 4.5 observamos el proceso. Primero el momentum calcula  la gradiente actual(vector azul pequeño)  y luego da un gran salto en la dirección de la gradiente actualizada acumulada (gran vector azul), el NAG primero realiza un gran salto en dirección del gradiente acumulado previo(vector marron) luego realiza un correción(vector rojo), esto nos da como resultado la actualización completa de NAG(vector verde). Esta actualización anticipada es muy importante debido a que nos impide ir demasiado rápido y mejora la capacidad de respuesta lo cual aumenta el rendimiento de las CNN.
\begin{figure}[H]
	\centering
	\includegraphics[width=0.5\textwidth]{Figures/nesterov.png}
	\caption{Convergencia Nesterov\\ Fuente:  \href{https://www.doc.ic.ac.uk/~js4416/163/website/neural-networks/optimisers.html}{\textit{www.doc.ic.ac.uk}}}
	\label{nesterov }
\end{figure}
\subsubsection{Adagrad}
Es una algoritmo optimización basada en la gradiente de descenso, el algoritmo adapta la tasa de aprendizaje  realizando actualizaciones más pequeñas para parámetros con carácteristicas que se repiten con más frencuencia y una tasa alta  para parámetros con carácteriscticas pocas frencuentes.
\subsubsection{Adam	}
\subsubsection{AdaMax}
\subsubsection{Nadam}
%\newpage
$\ $
%\thispagestyle{empty} % para que no se numere esta pagina
\chapter{Resultados}
En este capítulo discutiremos los resultados obtenidos durante el uso de optimizadores para acelerar el proceso de entrenamiento de red neuronal convolucional. Estos ejemplos fueron aplicados al dataset 

\section{Conclusiones}

\chapter{Conclusiones y Trabajo Futuro}
En este capítulo se describirán las conclusiones general que se encontraron al probar y estudiar los distintos métodos de optimización utilizados para el proceso de acelerar el proceso de entrenamiento de la red neuronal convolución.\\ Además se propondrán algunas mejoras para que el trabajo obtenga mejores resultados.


\section{Conclusiones}


\begin{itemize}

\item[•] Los métodos de optimización Adam y RMSprop obtuvieron los mejores resultados de precisión en ambas pruebas.
\item[•] A pesar de que el método de optimización Adam fue propuesto a partir del RMSprop. Adam fue superado en las pruebas realizadas.
\item[•] El método Adagrad fue el que menor precisión obtuvo en las pruebas. Esto se debió a su dificultad de trabajar con la suma de las gradientes al cuadrado lo cual poco a poco reduce su taza de aprendizaje.
\item[•] El RMSprop como una mejora del Adagrad, obtuvo mejores resultados que el adagrad. Esto debido a que trabajo con el promedio de la raíz de la gradiente anterior y tasas de decaimiento para controlar el problema de la disminución de la tasa de aprendizaje del Adagrad

\end{itemize}
Además de lo anterior ....

\section{Trabajo Futuro}
El propósito general de este seminario I fue adquirir el conocimiento y experiencia necesaria para poder trabajar con redes neuronales profundas. Los métodos de optimización fueron una manera de introducirme al área de las redes convolucionales y comprender las ventajas y desventajas de algunos métodos. \\
Los temas de aprendizaje automático y en particular del aprendizaje profundo son muy amplios y en este seminario se trato de acoplar los temas pero no se realizó un análisis más detallado debido a la amplitud del área. \\
En el Seminario II se trabajará con más detalle el campo de redes neuronales convolucionales, además se tratará de diseñar un red neuronal convolucional y comparar este modelo con algunos actualmente usados. Además de realizar un implementación más interactiva .\\ Este seminario fue limitado por la capacidad de la tarjeta gráfica usada debido a que en algunos ensayos la memoria era insuficiente para el futuro seminario se planea realizar las pruebas en mejores equipos como por ejemplo contratar servicios de maquinas virtuales de amazon u otro proveedor.



%\afterpage{\blankpage}\textsl{}
\bibliography{main}
\bibliographystyle{unsrt}
\afterpage{\blankpage}
\appendix
% Appendix A

\chapter{Título del apéndice} % Main appendix title

\label{AppendA} % For referencing this appendix elsewhere, use \ref{AppendixA}

Un ejemplo de los apendices







\end{document}