\chapter{Introducción}
%-- 10 lineas
En el campo de la inteligencia artificial las redes neuronales profundas tienen un papel muy importante, debido a que estas son el camino para que las computadoras realicen tareas que nuestros cerebros realizan de manera natural, tareas como el reconocimiento de voz, imágenes y patrones. En la actualidad empresas importantes utilizan las redes neuronales profundas, un ejemplo de esto es Google con el reconocimiento de voz e imágenes. Una característica de las redes neuronales profundas es que están compuestas por una gran cantidad de capas, lo cual dificulta el entrenamiento en computadoras que solo usan el CPU. Una manera de resolver este problema es mediante el uso de las GPU's debido que las tareas de entrenamiento son paralelizables. Además, se pueden utilizar GPU'S para acelerar el proceso de entrenamientos de nuestra red neuronal profunda. Por otro lado se necesitan métodos de optimización que junto a la fortaleza de las GPU's nos permitan obtener un mejor rendimiento.

\section{Motivación}
La inteligencia artificial constituye una base muy importante en el campo de la computación, mezcla un conjunto de disciplinas como la estadística y ciencia de la computación con el objetivo de construir modelos que puedan permitir a las computadoras realizar tareas que hace algunos años hubiesen sido considerada imposibles.\\
Actualmente las computadoras son capaces de reconocer objetos y clasificarlo. Esto ha permitido que la industria de la robótica se desarrolle de manera acelerada en las últimas décadas.
 Además hoy en día existen muchas herramientas que nos permiten desarrollar este tema y profundizarlo, pero a medida que aumenta la complejidad del problema, el costo computacional se incrementa, lo cual se convierte en un problema importante.

 Una de la soluciones que surgió fue el uso de la GPU's para acelerar procesos como el entrenamiento de una red neuronal con muchas capas ocultas, las GPU's representan una solución muy eficaz debido a que en el campo de la inteligencia artificial existen muchas tareas que son paralelizables.

Actualmente el mercado de GPU's evoluciona muy rápido debido a su gran demanda en la industria de los videojuegos, este mercado se encuentra dominado por NVIDIA y AMD, esta competencia y la alta demanda permite que las GPU's tengan mejor rendimiento lo cual puede ser usado para obtener mejores resultados en el campo del Aprendizaje Automático. 

Por otro lado, la optimización no solo se basa en el uso de hardwares más potentes, sino también depende de la elección de métodos adecuados para nuestros modelos, esta elección dependerá mucho del problema a tratar. Uno de los métodos más usado en el campo del Aprendizaje Automático es la gradiente de descenso estocástica pero este métodos por sí solo no es muy óptimo. 

Actualmente existe la problemática de hallar métodos más eficientes de optimización que obtengan un mejor rendimiento, el presente seminario se centra en la búsqueda y comparación de estos métodos con el fin de encontrar aquellos que sean más rápidos y eficientes. Además, adquirir el conocimiento y entender cómo es que estos funcionan.
%-----
%-----Que es lo que te ha motivado para realizar la tesis en esta temática y los aspectos más esenciales que pensabas obtener de ella...

%-----Este punto podrá ser de 4 páginas máximo. Es una de las partes más importantes que introduce al lector en el trabajo en tu tesis, por lo que debe estar muy bien redactado y estructurado.

\section{Objetivos}

El objetivo de este seminario es el de mostrar las ventajas del uso de distintos métodos de optimización para acelerar el entrenamiento de una red neuronal convolucional en una tarea de clasificación.

Específicamente, los objetivos de este trabajo con respecto al sistema son:

\begin{itemize}
\item[•] Entender el funcionamiento de las redes neuronales profundas.%--OBJETIVO ESPECÍFICO 1.

\item[•] Estudiar métodos de optimización en Aprendizaje Automático.

\item[•] Conocer las ventajas y desventajas de diferentes métodos de optimización.
\item[•] Mostrar los resultados de distintos métodos de optimización en el entrenamiento de una red neuronal convolucional para tareas de clasificación.



\end{itemize}

Y los objetivos con respecto a las competencias académicas desplegadas en el trabajo son:
\begin{itemize}
\item[•] Desarrollar un mejor entendimiento de las redes neuronales y sus aplicaciones, para así poder lograr afrontar problemas en el campo de la inteligencia artificial. %--OBJETIVO COMPETENCIA 1.
\item[•] Obtener la capacidad de discriminar entre los distintos métodos de optimización y elegir el adecuado para un problema de aprendizaje profundo.%--OBJETIVO COMPETENCIA 2.
\item[•] Obtener un conocimiento de las herramientas y recursos que existen actualmente para abordar problemas de aprendizaje profundo, además de poder analizar que herramientas son adecuadas para algunos problemas.
\end{itemize}

\section{Estructura del Seminario}


\begin{itemize}

\item \textbf{Introducción:} \\
En este capítulo introductorio se comenta sobre el tema a tratar, las motivaciones, intereses, objetivos con los cuales se planteo el presente seminario.
%--En este capítulo introductorio se comenta sobre las motivaciones ...

\item \textbf{Estado del Arte:} \\
Este capítulo muestra los trabajos e investigaciones ya realizadas, además de algunas aplicaciones que motivaron al presente seminario y además las investigaciones mostrarán el interés del problema planteado.

\item \textbf{Aprendizaje automático y Redes Neuronales:} \\
En este capítulo daremos una introducción general al Aprendizaje automático y distinguiremos los tipos de aprendizajes que existen, Además veremos los tipos de problema en esta área. Luego trataremos el tema de las Redes neuronales como una introducción al capítulo 4.
\item \textbf{Optimizadores para la gradiente de descenso en una Red Neuronal Convolucional:} \\
En este capítulo conoceremos más de un tipo específico de redes neuronal, las redes neuronales convolucionales. Detallares las principales diferencias con las redes tradicionales y describiremos sus principales hiperparámetros. Luego de eso nos enfocaremos en los optimizadores de la gradiente de descenso.
\item \textbf{Resultados:} \\
Se mostrarán los resultados obtenidos en las pruebas de los optimizadores además de describir los resultados.
\item \textbf{Conclusiones y Trabajo Futuro:} \\
En este capítulo se plantean las conclusiones y se detalla algunos inconvenientes encontrados durante el trabajo. Además que se comprueba la teoría descrita en el capítulo 4.
%--\item \textbf{Metodología y Herramientas:} \\
%--....

%--\item \textbf{...} \\
%--Un pequeño texto más de ese capítulo(s) en concreto.

%--\item \textbf{Conclusiones y Trabajo a Futuro:}\\
%--En este capítulo se exponen las conclusiones obtenididas de este trabajo. Adicionalmente, se proponen trabajos a futuro para la implementación de el servidor y la aplicación web del sistema.
\end{itemize}

%--NOTA: RECUERDE QUE ES COMO UN LIBRO TODO CAPÍTULO NUEVO COMENZARÁ CON PÁGINA IMPAR NUNCA PAR


