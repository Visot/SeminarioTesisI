\chapter{Introducción}
%-- 10 lineas
En el campo de la inteligencia artificial las redes neuronales profundas tienen un papel muy importante debido a que estas son el camino para que las computadoras realicen tareas que nuestros cerebros realizan de manera natural, tareas como el reconocimiento de voz, imagenes y patrones. En la actualidad empresas importantes utilizan las redes neuronales profundas uno ejemplo de esto es google con el reconocimiento de voz e imagenes. Un caracteristica de las redes neuronales profundas es que están compuestas por una gran cantidad de capas lo cual dificulta el entrenamiento en computadoras que solo usan el CPU. Una manera de resolver este problema es mediante el uso de las GPU's debido que las tareas de entrenamiento son paralelizables podemos usar las GPU'S para acelerar el proceso de entrenamientos de nuestra red neuronal profunda.

\section{Motivación}
La inteligencia artificial constituye una base muy importante en el campo de la computación, mezcla un conjunto de disciplinas como la estadística y ciencia de la computación con el objetivo de construir modelos que puedan permitir a las computadoras realizar tareas que hace años hubiese sido considerado imposible. El hecho de lograr que las computadoras sean capaces de reconocer objetos, clasificarlos lo cual ha permitido que la industria de la robotíca desarrolle de manera acelerada en las últimas décadas.  Hoy en día existen muchas herramientas que nos permiten desarrollar este tema y profundizarlo pero a medida que aumenta la complejidad del problema, el costo computacional incrementa lo cual se convierte un problema importante. Una de la soluciones que apareció fue el uso de la GPU's para acelerar procesos como el entrenamiento de una red neural con muchas capas ocultas, las GPU's representa una solución muy eficaz debido a que en el campo de la inteligencia artificial existen muchas tareas que son paralelizables.

 Actualmente el mercado de GPU's evoluciona muy rápido debido a su gran demanda en la industria de los videojuegos este mercado esta dominado por NVIDIA y AMD esta competencia y la alta demanda permite que las GPU's tengan mejor rendimiento, además actualmente la nube ofrece otra posibiliad para el desarrollo de GPGPU computing debido a que nos permite tener acceso a mejores recursos. La evolución de GPU's nos permite usarlos en para mejorar el procesamiento de grandes cantidades de datos.  
 
Las redes neuronales nos permiten realizar tareas 

%-----
%-----Que es lo que te ha motivado para realizar la tesis en esta temática y los aspectos más esenciales que pensabas obtener de ella...

%-----Este punto podrá ser de 4 páginas máximo. Es una de las partes más importantes que introduce al lector en el trabajo en tu tesis, por lo que debe estar muy bien redactado y estructurado.

\section{Objetivos}

El objetivo de este seminario es el de mostrar las ventanjas que presenta el uso de GPGPU computing frente a otro tipos de recursos en el entrenamiento de una deep neural network.

Especificamente, los objetivos de este trabajo con respecto al sistema son:

\begin{itemize}
\item[•] Entender el funcionamiento de las redes neuronales profundas%--OBJETIVO ESPECÍFICO 1.
\item[•] Mostrar los resultados de distintas GPU's en el entramiento de una red neuronal profunda.%--OBJETIVO ESPECÍFICO 2.
\item[•] Realizar un análisis comparar el rendimiento de cada GPU al momento de entrenar la deep neural network.%--OBJETIVO ESPECÍFICO 3.


\end{itemize}

Y los objetivos con respecto a las competencias académicas desplegadas en el trabajo son:
\begin{itemize}
\item[•] Desarrollar un manejo adecuado del uso GPU computing.%--OBJETIVO COMPETENCIA 1.
\item[•] Lograr un entendimiento de la tareas que son paralelizabeles en el entrenamiento de la red neuronal profunda.%--OBJETIVO COMPETENCIA 2.
\item[•] Desarrollar un manejo adecuado de tensorflow como herramienta para el reconocimiento de imagenes.%--OBJETIVO COMPETENCIA 3.

\end{itemize}

\section{Estructura del Seminario}


\begin{itemize}

\item \textbf{Introducción:} \\
En este capítulo introductorio se comentará sobres las motivaciones, objetivos con los cual se penso el presente seminario.
%--En este capítulo introductorio se comenta sobre las motivaciones ...

\item \textbf{Estado del Arte:} \\
En este capítulo observaremos los trabajos e investigaciones previa que tiene el presente seminario , además se expondrá el aporte que relizaremos.

%--\item \textbf{Metodología y Herramientas:} \\
%--....

%--\item \textbf{...} \\
%--Un pequeño texto más de ese capítulo(s) en concreto.

%--\item \textbf{Conclusiones y Trabajo a Futuro:}\\
%--En este capítulo se exponen las conclusiones obtenididas de este trabajo. Adicionalmente, se proponen trabajos a futuro para la implementación de el servidor y la aplicación web del sistema.
\end{itemize}

%--NOTA: RECUERDE QUE ES COMO UN LIBRO TODO CAPÍTULO NUEVO COMENZARÁ CON PÁGINA IMPAR NUNCA PAR


