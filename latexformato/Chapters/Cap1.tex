\chapter{Introducción}
%-- 10 lineas
En el campo de la inteligencia artificial las redes neuronales profundas tienen un papel muy importante debido a que estas son el camino para que las computadoras realicen tareas que nuestros cerebros realizan de manera natural, tareas como el reconocimiento de voz, imagenes y patrones. En la actualidad empresas importantes utilizan las redes neuronales profundas uno ejemplo de esto es google con el reconocimiento de voz e imagenes. Un caracteristica de las redes neuronales profundas es que están compuestas por una gran cantidad de capas lo cual dificulta el entrenamiento en computadoras que solo usan el CPU. Una manera de resolver este problema es mediante el uso de las GPU's debido que las tareas de entrenamiento son paralelizables podemos usar las GPU'S para acelerar el proceso de entrenamientos de nuestra red neuronal profunda.

\section{Motivación}
La inteligencia artificial constituye una base muy importante en el campo de la computación, mezcla un conjunto de disciplinas como la estadística y ciencia de la computación con el objetivo de construir modelos que puedan permitir a las computadoras realizar tareas que hace años hubiese sido considerado imposible. Hoy en día existen muchas herramientas que nos permiten desarrollar este tema y profundizarlo pero a medida que aumenta la complejidad del problema, el costo computacional incrementa lo cual se convierte un problema importante. Una de la soluciones que apareció fue el uso de la GPU's para acelerar procesos como el entrenamiento de una red neural con muchas capas ocultas, las GPU's representa una solución muy eficaz debido a que en el campo de la inteligencia artificial existen muchas tareas que son paralelizables.
El presente seminario fue pensado con el objetivos de mostrar y aprovechar las ventajas que ofrece el uso de GPU's para el entrenamiento de redes neuronales y 
%-----
%-----Que es lo que te ha motivado para realizar la tesis en esta temática y los aspectos más esenciales que pensabas obtener de ella...

%-----Este punto podrá ser de 4 páginas máximo. Es una de las partes más importantes que introduce al lector en el trabajo en tu tesis, por lo que debe estar muy bien redactado y estructurado.

\section{Objetivos}

El objetivo de este seminario es el de ... PONER OBJETIVO GENERAL .

Especificamente, los objetivos de este trabajo con respecto al sistema son:

\begin{itemize}
\item[•] OBJETIVO ESPECÍFICO 1.
\item[•] OBJETIVO ESPECÍFICO 2.
\item[•] OBJETIVO ESPECÍFICO 3.


\end{itemize}

Y los objetivos con respecto a las competencias académicas desplegadas en el trabajo son:
\begin{itemize}
\item[•] OBJETIVO COMPETENCIA 1.
\item[•] OBJETIVO COMPETENCIA 2.
\item[•] OBJETIVO COMPETENCIA 3.

\end{itemize}

\section{Estructura del Seminario}

Para brindrar al lector una idea global del contenido de este trabajo, a continuación se hace una breve descripción del propósito de cada capítulo presente en este seminario de tesis.
\begin{itemize}

\item \textbf{Introducción:} \\
En este capítulo introductorio se comenta sobre las motivaciones ...

\item \textbf{Estado del Arte:} \\
En este capítulo se establece el entorno ...

\item \textbf{Metodología y Herramientas:} \\
....

\item \textbf{...} \\
Un pequeño texto más de ese capítulo(s) en concreto.

\item \textbf{Conclusiones y Trabajo a Futuro:}\\
En este capítulo se exponen las conclusiones obtenididas de este trabajo. Adicionalmente, se proponen trabajos a futuro para la implementación de el servidor y la aplicación web del sistema.
\end{itemize}

NOTA: RECUERDE QUE ES COMO UN LIBRO TODO CAPÍTULO NUEVO COMENZARÁ CON PÁGINA IMPAR NUNCA PAR


